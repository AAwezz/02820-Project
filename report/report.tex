\documentclass[10pt]{IEEEtran}
\pdfoutput=1

\usepackage{graphicx}
\usepackage{hyperref}
\usepackage[utf8]{inputenc}
\usepackage{listings}
\usepackage[table]{xcolor}
\usepackage{pdfpages}

\hypersetup{colorlinks=true,citecolor=[rgb]{0,0.4,0}}


\title{Data mining with python: \\Automated FOREX trading}
\author{
	Kevin Voss Sjøbeck(s103451)\\
	\and
	Benjamin Maksuti(s103449)\\
	\and
	Anders Wessberg(s103477)
}


\begin{document}
\maketitle

\begin{abstract}
This project will utilize the oandapy API to get realtime data from the currency market to analyze, as well as enter and exit trades on based on the analytics. In the analytics we are going to use a short and a long moving averages, furthermore we are going to use MCAD as an extra precaution before we enter trades and use recovery zones to prevent losses on the account
\end{abstract}

\section{Introduction}

Lorem ipsum dolor sit amet, consectetur adipisicing elit, sed do
eiusmod tempor incididunt ut labore et dolore magna aliqua. Ut enim ad
minim veniam, quis nostrud exercitation ullamco laboris nisi ut
aliquip ex ea commodo consequat. Duis aute irure dolor in
reprehenderit in voluptate velit esse cillum dolore eu fugiat nulla
pariatur. Excepteur sint occaecat cupidatat non proident, sunt in
culpa qui officia deserunt mollit anim id est laborum. 

Lorem ipsum dolor sit amet, consectetur adipisicing elit, sed do
eiusmod tempor incididunt ut labore et dolore magna aliqua. Ut enim ad
minim veniam, quis nostrud exercitation ullamco laboris nisi ut
aliquip ex ea commodo consequat. Duis aute irure dolor in
reprehenderit in voluptate velit esse cillum dolore eu fugiat nulla
pariatur. Excepteur sint occaecat cupidatat non proident, sunt in
culpa qui officia deserunt mollit anim id est laborum.


\section{Getting data}
We obtain all the financial data of the financial instrument EUR vs USD on a 5 minute timeframe from oanda using their REST API for python. From oanda API we get the financial charts of the last 7 years from 2007-10-24 to 2014-10-24. After obtaining the data, it is stored in json fileformat in a file called "fxdata.txt", this allows us to use this data will then be used both to form our hypothosis of forex trading on the EUR vs USD currency pair, as well as performing a trading simulation on this set of data.

\section{Data Analasys}
To analyze the data we use a naive Bayes classifier, the idea to use a naive Bayes classifier on financial data, isn't new in fact, there are a couple of scientific articles on this already as can be seen in the article 

\section{Implementation}
This section will describe how we implemented a trading strategy based on our data analasys \cite{fx_naive_bayes}


\section{Research}
Article on FX Trading using Logistic Regression
$http://www.jsst.jp/e/JSST2012/extended_abstract/pdf/11.pdf$

\section{Terms \& abbrivations}

\begin{tabular}{l | l | l}
Domain & Term or Abbreviation &  Meaning  \\
Trading & FX & Forex \\ 
Trading & MA & Moving Average\\
Machine Learning & NBC & Naive Bayes Classifiers\\
\end{tabular}






\end{document}
